\documentclass[12pt, onecolumn]{article}
% Поддержка языков
\usepackage[english, russian]{babel} 

% Настройка кодировок
\usepackage[T2A]{fontenc}
\usepackage[utf8]{inputenc}

% Настройка шрифтов
\usepackage{fontspec}
\setmainfont[Ligatures=TeX]{Times New Roman} % Шрифт для основного текста документа

% Настройка отступов от краев страницы
\usepackage[left=5mm, top=20mm, right=5mm, bottom=10mm, nofoot]{geometry}

% пакеты для математики
\usepackage{amsmath, amsfonts, amssymb, amsthm, mathtools} 

\usepackage{tikz}
\usepackage{tkz-euclide}
\usetikzlibrary{angles,quotes}
%\usepackage{tempora} ломает русский язык
\usepackage{newtxmath}
\usepackage{graphicx}
\usepackage{wrapfig}
\usepackage{multicol}

\usepackage{fancyhdr} % пакет для установки колонтитулов
\pagestyle{fancy} % смена стиля оформления страниц
\fancyhf{} % очистка текущих значений

%%%%%%%%%%%%%%
%	Команды для линий и углов
%%%%%%%%%%%%%%

% линии
\newcommand{\boldline}[4]{
    \begin{tikzpicture}[scale=1.5, line width=2.5pt]
        \draw[#4, #3] (1,1) -- (1.8,1);
        \node[above] at (1,1) {\tiny #1};
        \node[above] at (1.8,1) {\tiny #2};
    \end{tikzpicture}
}

\newcommand{\thinline}[4]{
    \begin{tikzpicture}[scale=1.5, line width=1.2pt]
        \draw[#4, #3] (1,1) -- (1.8,1);
        
        \node[above] at (1,1) {\tiny #1};
        \node[above] at (1.8,1) {\tiny #2};
    \end{tikzpicture}
}

            
% углы

\newcommand{\BDC}{
    \begin{tikzpicture}[scale=1, line width=1.2pt]
          \draw pic[fill=red, angle radius=0.7cm] {angle = C--D--B};
	\filldraw[black] (d) circle (0pt) node[anchor=east]{\tiny D};
            \filldraw[black] (b) circle (0pt) node[anchor=west]{\tiny B};
            \filldraw[black] (c) circle (0pt) node[anchor=north]{\tiny C};    
    \end{tikzpicture}
}

\newcommand{\ACD}{
    \begin{tikzpicture}[scale=1, line width=1.2pt]
          \draw pic[fill=yellow, angle radius=0.7cm] {angle = a--c--d};
	\filldraw[black] (d) circle (0pt) node[anchor=east]{\tiny D};
            \filldraw[black] (a) circle (0pt) node[anchor=west]{\tiny A};
            \filldraw[black] (c) circle (0pt) node[anchor=west]{\tiny C};    
    \end{tikzpicture}
}

\newcommand{\ADC}{
    \begin{tikzpicture}[scale=1, line width=1.2pt]
	\draw pic[fill=blue, angle radius=0.7cm] {angle = b--d--a};
       \draw pic[fill=red, angle radius=0.7cm] {angle = c--d--b};
\filldraw[black] (d) circle (0pt) node[anchor=east]{\tiny D};
            \filldraw[black] (a) circle (0pt) node[anchor=south]{\tiny A};
            \filldraw[black] (c) circle (0pt) node[anchor=north]{\tiny C};    
    \end{tikzpicture}
}

\newcommand{\DCB}{
    \begin{tikzpicture}[scale=0.1, line width=1.2pt]
	\draw pic[fill=yellow, angle radius=0.5cm] {angle = a--c--d};
	\draw pic[fill=black, angle radius=0.5cm] {angle = b--c--a};
	 \filldraw[black] (d) circle (0pt) node[anchor=east]{\tiny D};
            \filldraw[black] (b) circle (0pt) node[anchor=west]{\tiny B};
            \filldraw[black] (c) circle (0pt) node[anchor=north]{\tiny C};    
    \end{tikzpicture}
}

\newcommand{\GEF}{
    \begin{tikzpicture}[scale=0.1]
	\draw pic[draw=black, line width = 1mm, dotted, angle radius=0.6cm] {angle = g--e--f};
       \draw[line width=0.5mm, red] (e) -- (g);
	\draw[line width=0.5mm, blue] (e) -- (f);
	\filldraw[black] (e) circle (0pt) node[anchor=south]{\tiny E};
            \filldraw[black] (f) circle (0pt) node[anchor=west]{\tiny F};
            \filldraw[black] (g) circle (0pt) node[anchor=east]{\tiny G};          
	
    \end{tikzpicture}
}

\newcommand{\CAB}{
    \begin{tikzpicture}[scale=0.1]	
	\draw pic[draw=black, line width = 1mm, dotted, angle radius=0.6cm] {angle = c--a--b};
	\draw[line width=1mm, blue] (a) -- (b);
     	\draw[line width=1mm, red] (c) -- (a);
	  \filldraw[black] (a) circle (1pt) node[anchor=south]{\tiny A};
            \filldraw[black] (b) circle (1pt) node[anchor=west]{\tiny B};
            \filldraw[black] (c) circle (1pt) node[anchor=east]{\tiny C};           
    \end{tikzpicture}
}
\newcommand{\DAB}{
    \begin{tikzpicture}[scale=0.1]	
	\draw pic[draw=red, line width = 1mm, angle radius=0.6cm] {angle = d--a--c};
	\draw pic[draw=black, line width = 1mm, dotted, angle radius=0.6cm] {angle = c--a--b};
	\draw[line width=1mm, blue] (a) -- (b);
     	\draw[line width=1mm, red] (c) -- (a);
	\draw[line width=1mm, red, dotted] (d) -- (a);
	  \filldraw[black] (a) circle (1pt) node[anchor=south]{\tiny A};
            \filldraw[black] (b) circle (1pt) node[anchor=west]{\tiny B};
            \filldraw[black] (d) circle (1pt) node[anchor=east]{\tiny D};           
    \end{tikzpicture}
}
%%%%%%%%%%%%%%%%%%
%	Основной	документ
%%%%%%%%%%%%%%%%%%

\date{}

\setlength{\columnsep}{-3cm}

\begin{document}

\fancyhead[C]{48} % установка верхнего колонтитула
\fancyhead[R]{КНИГА I ПРЕДЛ. XXIV. ТЕОРЕМА} % установка верхнего колонтитула
\renewcommand{\headrulewidth}{0pt} % убрать разделительную линию

 \begin{multicols}{2}    

	%	Чертёж пирамиды
        \begin{tikzpicture}[scale=0.8]       
	% Задаём точки 
         	\coordinate (A) at (5, 5);
            \coordinate (B) at (7, 0);
            \coordinate (C) at (3, -2);
            \coordinate (D) at (0, 0);
	     \coordinate (E) at (0, 0);
            \coordinate (F) at (2, -5);
            \coordinate (G) at (-3, -7); 
	\coordinate (a) at (0.5, 0.5);
        \coordinate (b) at (0.7, 0);
        \coordinate (c) at (0.3, -0.2);
	 \coordinate (d) at (0, 0);
	     \coordinate (e) at (0, 0);
            \coordinate (f) at (0.2, -0.5);
            \coordinate (g) at (-0.3, -0.7); 

            % Углы
		\draw pic[fill=blue, angle radius=1cm] {angle = B--D--A};
		\draw pic[fill=red, angle radius=1cm] {angle = C--D--B};
		\draw pic[fill=yellow, angle radius=1cm] {angle = A--C--D};
		\draw pic[fill=black, angle radius=1cm] {angle = B--C--A};
		\draw pic[draw=black, line width = 1mm, dotted, angle radius=1cm] {angle = C--A--B};
		\draw pic[draw=red, line width = 1mm, angle radius=1cm] {angle = D--A--C};
        
		% Линии
              \draw[line width=1mm, red, dotted] (D) -- (A);
		\draw[line width=1mm, blue, dotted] (D) -- (C);
	       \draw[line width=1mm, red] (C) -- (A);
 		\draw[line width=1mm, dotted] (C) -- (B);
		\draw[line width=1mm] (D) -- (B);
		\draw[line width=1mm, blue] (A) -- (B);
      
            % Подписываем точки
            \filldraw[black] (A) circle (1pt) node[anchor=south]{A};
            \filldraw[black] (B) circle (1pt) node[anchor=west]{B};
            \filldraw[black] (C) circle (1pt) node[anchor=north]{C};
            \filldraw[black] (D) circle (1pt) node[anchor=east]{D};         
        \end{tikzpicture}

	  \vspace{0.5cm}

	%	Чертёж треугольника
        \begin{tikzpicture}[scale=0.8]        
            
            % Углы
		\draw pic[draw=black, line width = 1mm, dotted, angle radius=1cm] {angle = G--E--F};
        
		% Линии
          	 \draw[line width=0.5mm, red] (E) -- (G);
		\draw[line width=0.5mm, blue] (E) -- (F);
	       \draw[line width=0.5mm, yellow] (G) -- (F);

            % Подписываем точки
            \filldraw[black] (E) circle (1pt) node[anchor=south]{E};
            \filldraw[black] (F) circle (1pt) node[anchor=west]{F};
            \filldraw[black] (G) circle (1pt) node[anchor=north]{G};          
        \end{tikzpicture}

        \vfill\null
        \columnbreak
	
	%	Основной текст
        \setlength{\columnsep}{-5cm}
        
        \begin{multicols}{2}
            \includegraphics[width=2.5cm, height=2.5cm ]{images/letter.png}
			
            \vfill\null
	
            \columnbreak
			
            \noindent сли \textit{у двух треугольников по две стороны соответсвенно равны друг другу (\boldline{A}{B}{blue}{} = \thinline{E}{F}{blue}{} и \boldline{A}{D}{red}{dashed} = \thinline{G}{E}{red}{}), и угол заключенный ними в одном \DAB больше, чем в другом \GEF, то сторона \boldline{D}{B}{black}{} противолежащая большему углу больше стороны, противолежайшей меньшему \thinline{F}{G}{yellow}{}.} 
        \end{multicols}
   	
  \vspace{1cm}

	\begin{center}
        Сделаем \CAB = \GEF (пр. I.$_{23}$),
 
        \vspace{0.5cm}
        
	и \boldline{C}{A}{red}{} = \thinline{G}{E}{red}{} (пр. I.$_3$),
	
	проведём \boldline{C}{D}{blue}{dashed} и \boldline{B}{C}{black}{dashed}. 

	\vspace{0.5cm}
        \par 
        
        Поскольку \boldline{C}{A}{red}{} = \boldline{A}{D}{red}{dashed} (акс. I, гип., постр.)

            $\therefore \ADC = \ACD $ (пр. I.$_5$), но \BDC < \ACD,
            
	   \vspace{0.5cm}

            и $\therefore$ \BDC < \DCB,

   	   \vspace{0.5cm}

	    $\therefore$ \boldline{D}{B}{black}{} > \boldline{B}{C}{black}{dashed} (пр. I.$_{19}$)

	   \vspace{0.5cm}

            но \boldline{B}{C}{black}{dashed} = \thinline{F}{G}{yellow}{} (пр. I.$._4$)

	   \vspace{0.5cm}

	    $\therefore$ \boldline{D}{B}{black}{} >\thinline{F}{G}{yellow}{}.
            
      
        
        \end{center}
        
        \begin{flushright}ч.т.д.\end{flushright}
        
    \end{multicols}
\end{document}
